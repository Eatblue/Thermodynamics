\documentclass[a4paper,12pt]{article}
\usepackage[left=1cm,right=1cm,top=3cm,bottom=3cm,a4paper]{geometry}
\usepackage{amsmath}
\usepackage[pdftex]{graphicx}
\usepackage{graphicx}
\usepackage{kotex}
\usepackage[onehalfspacing]{setspace}
\begin{document}
	\begin{flushleft}
		$<$Van der waals gas란? `*Reif 5.8절,5.9절$>$ 
	\end{flushleft}
\paragraph{Van der waals 기체의 내부에너지와 엔트로피}
Van der waals 기체 1몰의 상태방정식은 경험적으로 다음과 같이 나타낼 수 있다.
$$\left(p+\frac{a}{v^2} \right)(v-b)=RT  $$
여기서 $v$는 몰당 부피(부피/몰수)를 말한다. 이것을 van der waals 방정식이라고 부르고, van der waals 방정식을 쓰면 보다 실제적인 기체를 효과적으로 다룰 수 있다. 이상기체의 상태방정식과 다른 점은 $a,b$ 두 상수가 방정식 안에 존재하고 있다는 것이다. 실제 기체 분자는 분자간에 long-range attractive force가 있어서 넓은 범위 안에서 서로서로 붙어있으려고 노력한다. 이것은 마치 압력과 같은 효과를 유발하는데, 그게 방정식 안에서 $a$가 하는 일이다. 한편, 실제 기체 분자는 short-range repulsive force도 생각해주어야 한다. short-range replusive force는 서로 다른 기체 분자가 같은 공간을 차지하지 않게 해주는 힘이다. 따라서 기체 분자 각각이 고유한 부피를 차지하고 있으니 상태방정식을 기술할 때는 분자들의 부피를 빼주어야 한다. 그 역할을 바로 $b$가 한다.
\paragraph{}
$a=b=0$인 경우, 혹은 분자 1개의 부피가 아주 작아서 몰당 부피가 아주 클 때($v\rightarrow\infty$), van der waals 방정식은 다음과 같이 근사시킬 수 있다.
$$pv=RT$$ 
즉, 위와 같은 상황은 이상기체에 대한 기본가정을 설명한다.
\paragraph{}
Van der waals 기체의 몰당 에너지가 부피 의존성을 가지고 있을까? 몰당 에너지를 $u$, 몰당 엔트로피를 $s$라고 하면 열역학 1법칙에서,
$$du=Tds-pdv$$
$$\left. \frac{\partial u}{\partial v}\right) _{T}=T\left. \frac{\partial s}{\partial v}\right) _{T}-p$$
$$\begin{Bmatrix}
	du=Tds-pdv\\
	df=-sdT-pdv
\end{Bmatrix}\quad\rightarrow \partial_{T}(-p)|_{v}=\partial_{v}(-s)|_{T}$$ 
$$\left. \frac{\partial u}{\partial v}\right) _{T}=T\left. \frac{\partial p}{\partial T}\right) _{v}-p$$
Van der waals 방정식을 $p$에 대한 식으로 고치면,
$$p=\frac{RT}{v-b}-\frac{a}{v^2} \quad \longrightarrow \quad \left.\frac{\partial p}{\partial T}\right) _{v}=\frac{R}{v-b} $$
따라서,
$$\left. \frac{\partial u}{\partial v}\right) _{T}=T\left. \frac{\partial p}{\partial T}\right) _{v}-p=\frac{RT}{v-b}-p=\frac{a}{v^2}$$
이상기체는 $a$가 0이되어 에너지의 부피 의존성이 없다.($U=3/2k_BT$) 몰당 에너지 $u$를 다시 $T$ 와 $v$ 에 대한 함수로 나타내보자.
$$du(T,v)=\left.\frac{\partial u}{\partial T} \right)_{v}dT+\left.\frac{\partial u}{\partial v} \right)_{T}dv $$ 
$$=c_v(T)dT+\frac{a}{v^2}dv$$
이제 어떤 실제적인 기체가 초기에 온도 $T_0$와 몰당 부피 $v_0$를 가지고 있었다고 하자. 이 기체가 온도 $T$와 몰당 부피 $v$를 가질 때 에너지를 $u(T,v)$ 라고 하면,
$$u(T,v)-u(T_0,v_0)=\int_{T_0}^{T}c_v(T')dT'+\int_{v_0}^{v}\frac{a}{v^2}dv$$
$$=\int_{T_0}^{T}c_v(T')dT'-a\left(\frac{1}{v}-\frac{1}{v_0} \right) $$
$$\mbox{therefore, }\quad u(T,v)=\int_{T_0}^{T}c_v(T')dT'-\frac{a}{v}+const.$$
$c_v$가 상수라면,
$$u(T,v)=c_v T-\frac{a}{v}+const.$$
즉, 교과서 72쪽 4번째 줄부터 6번째 줄까지 "물리적으로는 기체가 팽창된 후에는 기체 분자 사이의 평균 거리가 팽창되기 전보다 더 멀어지므로, 분자 사이의 퍼텐셜에너지가 증가하게 된다"는 이러한 맥락에서 도출된 결과이다. $v$가 커질수록, $u$는 증가한다.
\paragraph{}
마지막으로 몰당 엔트로피 $s$를 계산해보자.
$$ds(T,V)=\left. \frac{\partial s}{\partial T}\right)_{v}dT+\left. \frac{\partial s}{\partial v}\right)_{T}dv $$
$c_v(T)=T\partial s/\partial T|_{v},\quad \partial_{T}(-p)|_{v}=\partial_{v}(-s)|_{T}$ 이므로,
$$ds(T,V)=\frac{c_v(T)}{T}dT+\left. \frac{\partial p}{\partial T}\right)_{v}dv$$
$$=\frac{c_v(T)}{T}dT+\frac{R}{v-b}dv$$
양변 적분하면,
$$s(T,v)-s(T_0,v_0)=\int_{T_0}^{T}\frac{c_v(T')dT'}{T'}+\int_{v_0}^{v}\frac{R}{v-b}dv$$
$$=\int_{T_0}^{T}\frac{c_v(T')dT'}{T'}+R\mbox{ ln}\left(\frac{v-b}{v_0-b} \right) $$
$ c_v$가 상수라면,
$$s(T,v)=c_v\mbox{ ln }T+R\mbox{ ln }(v-b)+const.$$
\paragraph{Van der waals 기체의 자유팽창} 위의 논의에서 우리는 Van der waals 기체의 몰당 에너지를 다음과 같이 나타낼 수 있다는 것을 알았다.
$$u(T,v)=\int_{T_0}^{T}c_v(T')dT'-\frac{a}{v}+const.$$
Free expansion 과정에서 주어진 판데르 발스 기체가 온도 $T_1$, 몰당 부피 $v_1$에서 온도 $T_2$, 몰당 부피 $v_2$로 팽창했다면($v_1<v_2$) 내부에너지가 보존되므로,
$$u(T_1,v_1)=u(T_2,v_2)$$
$$\int_{T_0}^{T_1}c_v(T')dT'-\frac{a}{v_1}=\int_{T_0}^{T_2}c_v(T')dT'-\frac{a}{v_2}$$
$$a\left(\frac{1}{v_2}-\frac{1}{v_1} \right)=\int_{T_1}^{T_2}c_v(T')dT' $$
$T_1,\,T_2$ 의 간격이 아주 짧다면 기체의 몰비열 $c_v$의 temperature dependence를 무시할 수 있다. 즉 기체의 몰비열을 상수 $c_v$로 취급할 수 있다.
$$a\left(\frac{1}{v_2}-\frac{1}{v_1} \right)=c_v(T_2-T_1) $$
$$T_2=T_1-\frac{a}{c_v}\left(\frac{1}{v_1}-\frac{1}{v_2} \right) $$
$v_2>v_1$에서 $(1/v_1-1/v_2)>0$ 이므로 $T_2<T_1$ 이다. 다시말해, 온도가 낮아졌다. 따라서 기체를 냉각할 때 자유팽창과정을 유용하게 쓸 수 있다. 하지만 우리가 간과한 사실이 있다. 지금까지 용기의 열용량을 애써 무시해왔는데 실제 실험에서 우리가 자유팽창과정을 이용해 기체를 냉각시키려면 용기의 열용량 $C_c$를 고려해야 하기 때문이다. 즉, 실제 실험에서 $T_2$는 
$$T_2=T_1-\frac{a}{c_v+C_c}\left(\frac{1}{v_1}-\frac{1}{v_2} \right) $$ 
으로 아까보다는 냉각효과가 많이 작아졌다.
\end{document}