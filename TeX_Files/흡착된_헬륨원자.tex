\documentclass[a4paper,12pt]{article}
\usepackage[left=1cm,right=1cm,top=3cm,bottom=3cm,a4paper]{geometry}
\usepackage[pdftex]{graphicx}
\usepackage{graphicx}\usepackage{amsmath}
\usepackage{amsfonts}
\usepackage{amssymb}
\usepackage{kotex}
\usepackage[onehalfspacing]{setspace}
\begin{document}
\paragraph{질량이 m인 헬륨 원자들은 금속의 표면에 흡착될 수 있으며, 흡착된 헬륨원자 하나를 금속 표면으로부터 떼어내기 위해서는 $\phi$ 만큼의 일이 필요하다. 2차원 금속 표면 위에 흡착된 헬륨 원자들은 서로 상호작용을 하지 않고 금속 표면을 자유롭게 움직일 수 있다고 가정하자. 만약 넓이가 $A$인 금속 표면이 압력이 $P$이고 부피가 $V$인 헬륨 기체와 접촉을 하고 전체 계가 온도 $T$에서 열적 평형 상태에 놓여 있다고 하자. 기체 상태의 헬륨 원자도 상호작용을 하지 않고 부피 $V$인 3차원 공간을 자유롭게 움직인다.} ($N_g$를 기체 헬륨 원자의 수, $N_s$를 금속 표면에 흡착된 헬륨 원자의 수라고 하자)
\begin{flushleft}
	(1) 위와 같이 구성된 계의 Partition function 을 구하자.\\
	$$\begin{cases}
	\mbox{흡착(2차원): }N_s&\mathcal{H}=-\phi,\quad Z_s(T,V,1)\\ \mbox{기체(3차원): }N_g&\mathcal{H}=\frac{p^2}{2m},\quad Z_g(T,V,1)
	\end{cases}$$
	* 입자 N개로 이루어진 계의 partition function *
	$$Z(T,V,N)=\frac{1}{N! h^3N}\int e^{_\beta \sum_iH_i}d^3qd^3p$$
	따라서 이 문제의 경우,
	$$Z(T,V,N)=Z_s(T,V,N_s)Z_g(T,V,N_g)$$이다. 각각을 구해보자.
	$$Z_s(T,V,N_s)=\frac{1}{N_s!}(Z_s(T,V,1))^{N_s}, \quad Z_g(T,V,N_g)=\frac{1}{N_g!}(Z_g(T,V,1))^{N_g}$$
	$$Z_s(T,V,1)=\frac{1}{h^2}\int e^{-\beta H}d^2qd^2p=\frac{1}{h^2}\int e^{\beta \phi}d^2qd^2p=\frac{A}{h^2}\int_{E\le \phi}d^p$$
	$p$에 대한 적분은 반지름이 $\sqrt{2m\phi}$ 인 2차원 원의 넓이와 같다. 따라서,
	$$Z_s(T,V,1)=\frac{2\pi m\phi A}{h^2}e^{\beta \phi}, \quad Z_s(T,V,N_s)=\frac{1}{N_s!}\left(\frac{2\pi m\phi A}{h^2}e^{\beta \phi} \right)^{N_s} $$ 
	기체 헬륨원자들은 이상기체로 간주하면 쉽게 풀린다. (수업시간에도 했다.)
	$$Z_g(T,V,1)=\frac{1}{h^3}\int e^{-\beta H}d^3qd^3p=\frac{1}{h^3}\int e^{-(\beta/2m) p^2}d^3qd^3p=\frac{V}{h^3}\left( \int_{\infty}^{\infty} e^{-(\beta/2m) p^2}d^3p\right)^3 $$
	따라서,
	$$Z_g(T,V,1)=\frac{V}{h^3}\sqrt{\frac{2m\pi}{\beta}}^3,\quad Z_g(T,V,N_g)=\frac{1}{N_g!}\left(\frac{V}{h^3}\sqrt{\frac{2m\pi}{\beta}}^3 \right)^{N_g} $$
	구한 것들을 합치면,
	$$Z(T,V,N)=Z_s(T,V,N_s)Z_g(T,V,N_g)=\frac{1}{N_s!}\left(\frac{2\pi m\phi A}{h^2}e^{\beta \phi} \right)^{N_s}\frac{1}{N_g!}\left(\frac{V}{h^3}\sqrt{\frac{2\pi m}{\beta}}^3 \right)^{N_g}$$
	로그를 취하면 다음과 같다. (stirling formula: $\ln N!=N\ln N-N$)
	$$\ln Z=-\ln N_s!+N_s\left[\ln\left(\frac{2\pi m A \phi}{h^2} \right)+\beta \phi  \right]-\ln N_g!+N_g\left[\ln\left( \frac{V}{h^3}\right)+\frac{3}{2}\ln\left(\frac{2\pi m}{\beta} \right)   \right]  $$
	$$=N_s\left[\ln\left(\frac{2\pi m A \phi}{h^2} \right)+\beta \phi+1-\ln N_s  \right]+N_g\left[\ln\left( \frac{V}{h^3}\right)+\frac{3}{2}\ln\left(\frac{2\pi m}{\beta} \right) +1-\ln N_g  \right]$$
	총 에너지를 구해서 답이 맞나 확인해보자. $N_s$개가 흡착되어있고($-\phi$), $N_g$개가 이상기체 이므로($\frac{3}{2}k_BT$) 나와야 하는 에너지 값은 $\frac{3}{2}N_gk_BT-N_s\phi$ 이다.
	$$U=-\frac{\partial \ln Z}{\partial \beta}=-\left(N_s \phi+\frac{3}{2}N_g\left(\frac{-1}{\beta} \right)  \right)=\frac{3}{2}N_gk_BT-N_s\phi $$
	맞는거 같다.. 
\end{flushleft} 
\begin{flushleft}
	(2) partition function을 이용해서 Free energy 구하기
	$$ F=-k_B T(\ln Z)$$
	$$=-k_BT\left(N_s\left[\ln\left(\frac{2\pi m A \phi}{h^2} \right)+\beta \phi+1-\ln N_s  \right]+N_g\left[\ln\left( \frac{V}{h^3}\right)+\frac{3}{2}\ln\left(\frac{2\pi m}{\beta} \right) +1-\ln N_g  \right] \right) $$
		$$=-k_BT\left(N_s\left[\ln\left(\frac{2\pi m A \phi}{N_s h^2} \right)+\beta \phi+1  \right]+N_g\left[\ln\left( \frac{V}{N_g h^3}\right)+\frac{3}{2}\ln\left(\frac{2\pi m}{\beta} \right) +1 \right] \right) $$
		$$=-k_BT\left(N_s\left[\ln\left(\frac{2\pi m A \phi}{N_s h^2} \right)+\beta \phi+1  \right]+N_g\left[\ln\left( \frac{V}{N_g h^3}\left(\frac{2\pi m}{\beta} \right)^{3/2}\right)+1 \right] \right) $$
			$$=-k_BT\left(N_s\left[\ln\left(\frac{2\pi m A \phi}{N_s h^2} \right)+\beta \phi+1  \right]+N_g\left[\ln\left( \frac{V}{N_g \lambda_T^3}\right)+1 \right] \right) $$
	(3) 자유에너지로부터 금속표면에 단위 면적당 흡착된 헬륨 원자의 수를 구하시오(: 앞에서 $N_s$로 줘 놓고 왜 물어보는지 모르겠음)
	상태방정식을 구해보자.
	$$P=-\left( \frac{\partial F}{\partial V}\right)_{T,N}=k_BT\left( \frac{N_g}{V}\right)  $$
	$$PV=N_gk_BT$$
	$N_g$가 흡착이 되지않고 기체상태로 남아있으므로 나머지인 $N_s$가 흡착이 되어있다고 볼 수 있다. 따라서 단위면적당 흡착 원자 수는 $N_s/A$\\
	(4) 엔트로피
	$$S=-\left(\frac{\partial F}{\partial T} \right)_{V,N}\equiv \frac{U-F}{T} $$
	$$U-F\rightarrow\left(\frac{3}{2}N_gk_BT-N_s\phi \right) +k_BT\left(N_s\left[\ln\left(\frac{2\pi m A \phi}{N_s h^2} \right)+\beta \phi+1  \right]+N_g\left[\ln\left( \frac{V}{N_g \lambda_T^3}\right)+1 \right]  \right) $$
	$$= +k_BT\left(N_s\left[\ln\left(\frac{2\pi m A \phi}{N_s h^2} \right)+1  \right]+N_g\left[\ln\left( \frac{V}{N_g \lambda_T^3}\right)+\frac{5}{2} \right]  \right)$$
	$$S=k_B\left(N_s\left[\ln\left(\frac{2\pi m A \phi}{N_s h^2} \right)+1  \right]+N_g\left[\ln\left( \frac{V}{N_g \lambda_T^3}\right)+\frac{5}{2} \right]  \right)$$
	
\end{flushleft}
\end{document}